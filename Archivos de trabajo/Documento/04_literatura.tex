\chapter{Literatura}\label{chap2} 


Tochklov (2020) desarrollaron un ITCRM (lo llaman RER en base a las siglas) para 77 regiones de Rusia, contrastándo su evolución con la del índice nacional entre 2000 y 2016. 

La definición que proponen para RER es la que conocemos:
{
RER is defined as the nominal bilateral exchange rate
adjusted for the relative price between the home country and its trading partner. RER can be expressed
as a bilateral exchange rate, but given that countries have numerous trading partners, their corresponding
bilateral RERs are usually averaged to generate a weighted index.
}

Se plantea la cuestión acerca de si hacer un promedio geométrico o aritmético, y se concluye que el geométrico es mejor, ya que el aritmético puede sobreestimar la variabilidad del índice. Al respecto, leer:

\begin{itemize}
    \item \citet{brodsky_1982}. Arithmetic versus Geometric Effective Exchange Rates. Federal Reserve Bank of St. Louis Review, 64(5), 3–12.
    \item Rosensweig, J. (1987). Exchange rate index construction: With a US dollar application. Journal of Foreign Exchange and International Finance, 1(3), 293–301. [NO LO PUDE ENCONTRAR]
\end{itemize}

Al respecto del índice de precios a escoger, Tochklov (2020) sugiere índice de precios mayoristas: "The selection of price indices is also of major importance for RER. Conceptually, Producer Price Index (PPI) is the preferable option given the focus on competitiveness of suppliers and the fact that the
Consumer Price Index (CPI) contains a substantial number of nontraded goods and services, a large import component, and products with controlled or subsidies prices (Clark et al., 1999; Ellis, 2001; Rosensweig, 1987)". No obstante, reconocen que muchos bancos centrales e instituciones que producen ITCRM, utilizan el CPI dada la disponibilidad (At the same time, many central banks and international organizations employ the CPI as it is readily available over a long period and is comparable across a large number of countries (see Appendix II in Klau \& Fung, 2006; Table 3 in Lauro \& Schmitz, 2012; Ellis, 2001)) y dejan abierta la posibilidad de incluir otros tipos de deflactores (Alternative deflators that have been used in the literature include unit labor costs and the GDP deflator, but their coverage across countries and time is limited.)

En suma, siguen la metodología de Clark et al. (1999) según la cual el índice se calcula como:

\[
RER_{j,t} = \prod_{j=1}^{N} \left( \frac{E_{t}P_{it}}{P_{jt}} \right) / \left(  \frac{E_{b}P_{ib}}{P_{jb}} \right)^{w_j}
\]

donde $E_{t}$ es el tipo de cambio nominal expresado en términos de la moneda del país $j$ expresado en pesos en el momento $t$, $P_{it} y P_{jt}$ son los índices de precios para la región $i$ y el país $j$ en el momento $t$, respectivamente, $w_j$ es el peso asignado al país $j$, y el subíndice $b$ denota el período base.

[Nota: se podría armar un ITCRM regional en base a las regiones del IPC de INDEC. Esta formula no nos sirve exactamente para la provincia de Santa Fe. Me parece que una mejor opción es el paper de Yan et al. (2016).]

La selección de socios comerciales:

- En el caso de Tochklov (2021): aquellos países que contabilizan un total del 85\% de las exportaciones e importaciones a nivel nacional durante el período 2000-2016. The weights are fixed for the entire sample period based on the trade levels in 2014, which is the earliest
year with foreign trade data available for the complete set of Russian regions as reported by the
Federal Customs Service.

- En el caso de Yan et al. (2016): While including all trading partners would be an ideal approach, it is not viable. The
Real effective Exchange Rate and Regional Economic Growth in China 47 ©2016 Institute of World Economics and Politics, Chinese Academy of Social Sciences IMF practice includes the top 20 trading partners according to bilateral trading volume.
Instead, we intend to choose as many leading trading partners as possible to ensure the accuracy of the REER indices. In this paper, we select the top 40 trading partners to cover at least 80 percent of trading volume of the province. The trading partners included in the REER construction are listed in Table 1.

- En el caso de Clar et al (1999), fueron por importacia de exportaciones, incluyendo un total de 50 países que alcanzan el 91\%.

No hay muchísima bibliografía aparentemente de tipos de cambio regionales... buena noticia, porque podemos hacer algo novedoso.


