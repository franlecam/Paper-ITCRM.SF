% =============================
% Idioma y matemáticas
% =============================
\usepackage[spanish, es-lcroman]{babel} 
\usepackage{amsmath,amsfonts,amssymb}

% =============================
% Tu estilo para paper
% =============================
\usepackage{paper} % debe estar después de los paquetes base

% =============================
% Otros paquetes
% =============================
\usepackage{xcolor}       % Colores para comentarios
\usepackage{csquotes}     % Citas
\usepackage{graphicx}     % Para imágenes
\usepackage{caption}      % Configuración de captions
\usepackage{subcaption}   % Subfiguras
\usepackage{pdfpages}     % Para incluir PDFs
\usepackage[refpage]{nomencl} % Nomenclatura

% =============================
% Tablas
% =============================
\usepackage{multicol}  % Múltiples columnas
\usepackage{multirow}  % Múltiples filas
\usepackage{booktabs}  % Tablas más bonitas

% =============================
% Referencias
% =============================
\usepackage{hyperref}  % Links y DOI
\usepackage[style=apa, backend=biber]{biblatex} % APA 7
\addbibresource{bibliografia.bib} % tu archivo .bib

% =============================
% Texto de relleno para pruebas
% =============================
\usepackage{lipsum}

% =============================
% Nomenclatura
% =============================
\makenomenclature
\usepackage{ifthen}
\renewcommand{\nomgroup}[1]{%
	\ifthenelse{\equal{#1}{A}}{\item[\textbf{Acronyms}]}{%
		\ifthenelse{\equal{#1}{S}}{\item[\textbf{Symbols}]}{}}}
\renewcommand*\pagedeclaration[1]{,~\textit{p.\,\hyperpage{#1}}}

% =============================
% Configuración de captions
% =============================
\captionsetup{labelsep=space, labelfont=bf}
\renewcommand{\thefigure}{\arabic{chapter}.\arabic{figure}}
\renewcommand{\thetable}{\arabic{chapter}.\arabic{table}}

% =============================
% Título, autor y abstract (definir en main.tex)
% =============================
\title{Título del paper}
\author{Nombre de Autor}
\newcommand{\abstracttext}{Resumen del paper aquí. \lipsum[1]}

\makeatletter
\let\@abstract\abstracttext
\makeatother